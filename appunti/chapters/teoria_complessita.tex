

\section{Categorie}

\dots

\subsubsection{Problemi intrattabili}

Sono problemi per cui non esiste né un algoritmo $A_{\bpi{}}$ che li risolva, né un lower bound esponenziale.
Tuttavia c'è forte evidenza che il \textit{lower bound} esista, infatti si può provare che tutti i problemi in questa classe sono equivalenti, e questa proprietà di chiusura spinge verso la loro difficoltà.

\section{Problemi Decisionali}

\subsection{Problemi di ottimizzazione}

In un problema di ottimizzazione, si definisce una funzione di costo associato ad una soluzione:
\begin{equation*}
    c : \bs{} \rightarrow \mathbb{R}
\end{equation*}
e, dato l'insieme di soluzioni ammissibili associate ad un'istanza
\begin{equation*}
    \bs{} (i) = \left\{ s \in \bs{} : i \, \bpi{} \, s \right\}
\end{equation*}
si vuole individuare la soluzione di costo massimo (o minimo)
\begin{equation*}
    s_{i}^{*} = \argmax \left\{ c(s) : s \in \bs{} (i) \right\}
\end{equation*}

\subsection{Problemi Decisionali}

Per definire problemi e complessità degli algoritmi in modo rigoroso è necessario standardizzare i problemi computazionali.

Un problema decisionale presenta un insieme delle soluzioni ridotto a due elementi, ed associa ad un'istanza generica una risposta (positiva o negativa) ad una domanda sull'istanza.
\begin{equation*}
    \bpi{}_{D} : \bi{} \rightarrow \left\{ \text{Sì, No} \right\}
\end{equation*}
La restrizione a questo insieme di soluzioni non comporta una perdita di generalità, per esempio, si consideri il problema di trovare un cammino minimo tra due nodi in un grafo:

\subsubsection{\textit{Shortest Unweighted Path}}

% Il problema è definito sull'istanza $<G=(V,E), s, t>$, dove $V \subseteq \mathbb{N}$ finito (i nodi sono identificati con numeri naturali), $E \subseteq V \times V$ (gli archi sono coppie ordinate di vertici), $s, t \in V$. L'insieme delle soluzioni è $\mathbb{N}^*$, e una soluzione generica è $s=<v_1, \dots, v_k$, dove $v_i \in V, v_1=s, v_k=t$ e $\left( v_i, v_{i+1} \right) \in E, 1 \leq i < k$.
\begin{align*}
    \text{istanza:} & <G=(V,E), s, t> \\
    & V \subseteq \mathbb{N} \text{finito (i nodi sono identificati con numeri naturali)}
\end{align*}
    $E \subseteq V \times V$ (gli archi sono coppie ordinate di vertici)
    $s, t \in V$
    soluzioni: $\mathbb{N}^*$, e una soluzione generica è $s=<v_1, \dots, v_k$
    dove $v_i \in V, v_1=s, v_k=t$ 
    $\left( v_i, v_{i+1} \right) \in E, 1 \leq i < k$.










\section{Pezzi utili di \LaTeX{}}
\begin{algorithm}[H]
\caption{Divide and Conquer}\label{alg:dnc}
\begin{algorithmic}[1]
    \Procedure{D\&C}{$i$}
        \If{$|i| \leq n_0$}                             \Comment{BASE}
            \State *risolvo direttamente*
        \EndIf
        \State $<i_1, i_2, \dots, i_k> \gets A_D(i)$    \Comment{DIVIDE}
        \For{$j \gets 1 $ to $ k $ }                    \Comment{RECURSE}
            \State $s_j \gets $ \Call{D\&C}{$i_j$}
        \EndFor
        \State $s \gets A_C(<s_1, s_2, \dots, s_k>)$    \Comment{CONQUER}
        \State return $s$
    \EndProcedure
\end{algorithmic}
\end{algorithm}


